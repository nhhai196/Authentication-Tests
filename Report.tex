% 12 point font, and your thesis is a ``report'' to LaTeX
\documentclass[12pt]{report}

% this enables correct linespacing and graphics inclusion via 
%``\includegraphics''
\usepackage{setspace}
\usepackage{graphicx}
\usepackage{amssymb}
\usepackage{coqdoc}


% leave 1.5in margin to the left and 1in margin to the other
% sides. Don't print page number in the margin (but rather above it)
\setlength{\textheight}{8.63in}
\setlength{\textwidth}{5.9in}
\setlength{\topmargin}{-0.2in}
\setlength{\oddsidemargin}{0.3in} 
\setlength{\evensidemargin}{0.3in}
\setlength{\headsep}{0.0in}
\setlength{\parskip}{0.5em}

% Start to write
\begin{document}

\newcommand{\brk}{\vspace*{0.18in}}

% No page number on the title page
\thispagestyle{empty}

% Center the whole title page
\begin{center}

\brk
   {\large 
	\textbf{
	 	A Formalization of Strand Spaces in Coq
	}
   }


\brk
by

\brk
% insert your name here. 
Hai Nguyen


% All this is constant:
\brk\brk
A Major Qualifying Project

\brk
Submitted to the Faculty

\brk
of the 

\brk
WORCESTER POLYTECHNIC INSTITUTE
	
\brk
In partial fulfillment of the requirements for the

\brk
Degree of Bachelor of Science

\brk
in

\brk
Computer Science and Mathematical Sciences

\brk
by

% This is how LaTeX draws lines :) It's where your signature goes.
\brk\brk
\rule{3in}{1.2pt}

% Adjust this to your preferred month and year
\brk
Jan 2015

\end{center}

	
\vfill
APPROVED:

\vspace{0.5in}
\rule{3in}{0.8pt}

% Change this 
Professor Daniel J. Dougherty, MQP Advisor

% end of titlepage
\newpage

\doublespacing

\begin{abstract}
  In a cryptographic protocol, suppose a principal creates and
  transmits a message containing a new value $v$ which later receives
  back in different cryptographic context. We can conclude that some
  principal possessing the relevant key has received and transformed
  the message containing $v$. In many cases, the action of receiving
  and transforming must be done by some regular principal not a
  penetrator. An inference of such argument is an authentication test.
  With such tests, we can determine whether a certain value like a
  nonce remains secret so we can check whether certain security
  properties are  achieved in a cryptographic protocol.
  % 
  In this paper we formally prove the correctness of two
  authentication tests under the strand space formalism approach by
  using the Coq proof assistant. Coq is a formal proof management
  system. It provides a formal language to express mathematical
  assertions, mechanically checks proofs of these assertions. It also
  helps to find formal proofs and extracts a certified program from
  the constructive proof of its formal specification. Coq works within
  the theory of the calculus of inductive constructions, which is a
  variation on the calculus of constructions. We first formalize
  strand spaces by giving definitions in Coq of the basic notions.
  Then we express the two authentication tests and give constructive
  proofs for them. \cite{Guttman, chlipala2010introduction, bertot2004coq, ss}

\end{abstract}

\pagenumbering{roman} % or {Roman} if you like them capitalized

%
\begin{center}
	\textbf{Acknowledgements}
\end{center}

	I would like to express my gratitude to my advisor who made
	sure my MQP has at least 40 pages, 20 pictures and lots
	of formulas and thus made me master \LaTeX{} like my native
	language.

        My thanks are also due to my reader... who has read the thesis
        in the two days that I gave him since it wasn't done until two
        days before due date.

	Thanks also to ... lots of friends, the fact that a week has
	seven days instead of only five as I had always thought, and
	the fact that I own a key to the building so I can work at four
	in the morning whenever I feel like it. That is, all the time.

\clearpage
\tableofcontents

\listoffigures
\listoftables

\clearpage

\pagenumbering{arabic}
\setcounter{page}{1}

\chapter{Introduction.}
\section{Objectives and Project Motivation.}
Cryptographic protocols are intended to let principals communicate securely over a communication protocol which are designed to provide various kinds of security assurances. An important security goal of cryptographic protocol is authentication, the act of confirming the truth of an attribute of a datum or entity like verifying freshness of a nonce. Many research papers about authentication have published. One of them is “Authentication Tests and the Structure of Bundles” by Joshua Guttman and Javier Thayer. The main idea of authentication tests is that if a principal in a cryptographic protocol creates and transmits a message containing a new value v, and later receives v back in a different cryptographic context then it can be concluded that some principal processing the relevant key has received and transformed the message in which  was emitted. The authentication tests themselves are easy to apply but the proof justifying them are more complicated. Though authentication tests are proved in the paper, they have not been formally verified.  As we know that once lemma or a theorem has been proved in some proof assistant language like Coq, we will have a very strong assurance that it is true - much more than what we usually have when doing a pen-and-paper proof.\\
In addition, we found that there are few papers and projects using Coq to verify security goal of cryptographic protocols and to particularly formalize strand spaces, which is a well-known approach to cryptographic protocols.\\
In this project we will prove authentication tests under strand space formalism approach by using the Coq proof assistant. First, we will formalize strand spaces and all stuffs needed for proving authentications tests like components,transformation paths, penetrable keys, and so on... Then, we will give detailed formal proofs of all relevant lemmas, theorems, and finally authentication tests.\\
This project will help researchers in security area have more confidence in using the result of authentication tests since they are formally verified. Our implementation is modular so that researchers can easily extract certain modules for their purpose. For example, the formalization of strand spaces can be used as a frame work for later research using strand space approach.  
\section{Strand Space Overview.}
In this section, we briefly summarize the ideas behind the strand space model. The Coq development in the next chapter will provides precise definitions.\\
A strand spaces is a set of strands; one may think of a strand space as containing all legitimate executions together with all the actions that a penetrator may apply to the messages contained in these executions.\\
A strand is a sequence of events that a single principal, either a legitimate principal or a penetrator, may engage in. The height of a strand is the number of nodes on that strand. Each strand is a sequence of message transmissions and receptions with specific values such as nonces and keys. Transmission of a term $t$ is represented as $+t$ and reception of a term $t$ is represented as $-t$. Each element of a strand is called a node. Given a strand $s$, $(s,i)$ is the $i^th$ node on $s$. We say that $n \Rightarrow n'$ if $n=(s,i)$ and $n'=(s,i+1)$. Thus, the relation $\Rightarrow^+$ between two nodes is the transitive closure of the relation $\Rightarrow$. The relation $n \rightarrow n'$ represents the inter-strand communication; it means that $term(n)=+t$ and $term(n')=-t$; here $term(n)$ denotes the signed (unsigned) message at the node $n$.\\
Let $A$ be the set of all possible messages that can be exchanged between principals in a protocol. We call elements of $A$ terms. $A$ is freely generated from two disjoint sets, set of texts $T$ and set of cryptographic keys $K$, by concatenations $encr : K \times A \rightarrow A$ and encryptions $join : A \times A \rightarrow A$. Hence, $A$ is closed under concatenation and encryption. The set $K$ is equipped with an injective unary operator $inv : K \rightarrow K$ which maps each member of asymmetric key pair to the other and maps a symmetric key to itself.\\
A signed term is a pair of a sign $\sigma \in {+,-}$ and a term t, written either $<\sigma,t>$ or $+t$ or $-t$.\\
A term $t_1$ is a subterm of another term $t_2$, denoted as $t_1 \sqsubset t_2$, if we can get $t_2$ from $t_1$ by repeatedly concatenating with arbitrary terms and encrypting with arbitrary keys. For example, $A, N_a$ are subterms of ${|N_aA|}_K$ but $K$ is not.\\
Another important concept under strand space is origination. We say that a term t originates at a node n if n is a transmission node, $t \sqsubset term(n)$, and t is not a sub-term of any earlier node of $n$; hence, $n$ is the first node in its strand includes $t$. A node is called uniquely originating if it is originated on only one node over all strands. \\
A bundle is a casually well-founded collection of nodes and two relations $\Rightarrow$ and $\rightarrow$. It represents the actual protocol interactions. In a bundle, when a a strand receives a message $m$, there is a unique node transmitting $m$ from which the message was immediately  received. In contrast, when a strand transmits a message $m$, many strands or none may immediately receive $m$. The height of a strand in a bundle is the number of nodes on the strand that are in the bundle.\\
The penetrator's powers are characterized by the set of compromised keys which are initially known to penetrator, and a set of penetrator strands that allow the penetrator to generate new messages. The set of compromised keys typically would contain all public keys, all private keys of penetrators, and all symmetric keys initially shared between the penetrator and principals playing by the protocol rules. The atomic actions available to penetrator are encoded in a set of penetrator strands. We partition penetrator strands according to the operations they exemplify. E-strands encrypt when given a key and a plain-text; D-strands decrypt when given a decryption key and matching cipher-text; C-strands concatenate terms; S-strands separate terms; M-strands emit known atomic text or guess; and K-strands emit keys from a set of known keys.\\
Important units for protocol correctness are components. A term $t$ is a component of another term $t'$ if $t \sqsubset t'$, t is nt a concatenated term, and for every $s \neq t$ such that $t \sqsubset s \sqsubset t'$, $s$ is a concatenated term. Thus, a component is either atomic value or an encryption. A term $t$ is new at a node $n=<s,i>$ if t is a component of $term(n)$ but $t$ is not a component of node $<s,j>$ for every $j < i$. A component is new even if it has occurred earlier as a nested subterm of some larger component. When a component occurs new in a regular node but was a subterm of some previous node, then the principal executing that strand has done some cryptographic work to extract it as a new component. 
\section{The Coq Proof Assistant Overview.}
\subsection{What is Coq?}
We briefly describe what the Coq proof assistant is in this section.\\
Under programming language point of view, Coq implements dependently typed functional programming language, while under logical system, it implements a higher-order type theory. Coq exploits the notion of Curry-Howard isomorphism - the correspondence between proofs and programs. The relation between a proof and the statement it proves is the same as the relation between a program and its type. At the level of proofs and programs, we have the following correspondence:
\begin{center}
	\begin{tabular}{|c|c|}
		\hline 
		Logic side & Programming side \\ \hline
		hypothesis & free variables \\ \hline
		implication elimination & application \\ \hline
		implication introduction & abstraction \\ \hline
	\end{tabular}
\end{center}
While the below table summaries the correspondence at the level of terms and types:
\begin{center}
	\begin{tabular}{|c|c|}
		\hline
			Logic side & Programming side \\ \hline 
			universal quantification & generalised function space \\ \hline
			existential quantification & generalised cartesian product \\ \hline
			implication	& function type \\ \hline
			conjunction	& product type \\ \hline
			disjunction	& sum type \\ \hline
			true formula & unit type \\ \hline
			false formula & bottom (empty) type \\ \hline
	\end{tabular}
\end{center}
The correspondence says that implication behaves the same as a function type, conjunction as product type, disjunction as sum type, and so on... For example, $T:\tau$ means a term $T$ of type $\tau$ or equipvalently a proof $T$ of the proposition $\tau$; $A \rightarrow B$ is the type of a function that associate a term of type $B$ to any term of type $A$, while a proof of $A \rightarrow B$ is a term of that type or a term of the form $\lambda x.t$ where x is a proof of $A$ and t is a proof of $B$. \\
The coq system is based on the Calculus of Inductive Constructions (CIC) - from version V8 it is based on a weaker calculus, namely Predicate Calculus of Inductive Constructions (pCIC). There is usually a syntatic distinction between types and terms in most type theories. However, types and terms are defined as the same syntatic structure so everything even type is a term in Coq. Consequently, all objects have a type - atomic types, types for functions, types for proofs, types for types. When manipulated as terms, types are themselves a type which is a constant of the language called a sort. Prop and Set are the two base sorts. The sort Prop is the universe of propositions. The sort Set intends to be the type of small sets and includes data types such as booleans, natural numbers, and but also includes products, subsets, function type over these data types. The language of CIC also has typed terms, conversion rules, derived rules, and (co)inductive definitions.\\ 
The Coq system is a computer tool for mechanically verifying theorem proofs, which means that once you have proved something in Coq, you have strong assurance that it is true - more than what you usually have when doing a pen-and-paper proof. These theorems may concern usual mathematics, proof theory, or program verification. The Coq proof assistant are very powerful and expressive both for reasoning and programming. We can construct from simple terms and write simple proofs to building whole theories and complex algorithms. It provides an environment for defining objects (integers, sets, trees, functions...), making statements using logical connectives and basic predicates, and writing proofs. It also provides program extraction towards Haskell and Ocaml for efficient execution of algorithms and linking with other libraries.\\
The Coq compiler automatically checks the correctness of definitions (well-formed
sets, terminating functions...) and of proofs.\\
Coq is a proof assistant similar to higher order logic (HOL) systems, a family of interactive theorem prover based on Church's HOL including Isabelle, PVS... Unlike these systems, Coq is based on intuitionistic type theory. Consequently, it is closer to Epigram, and NuPrl... The common properties of these system are that functions are programs that can be computed and not just binary relation.
Coq can be used from standard teletype-like shell window but preferably through the graphical user interface called CoqIde. Coq is not an automated theorem prover which means that it does not automatically prove theorems. However, it can be considered as a semi-automated theorem prover since it includes many automatic theorem proving tatics and various decision procedures. It greatly simplifies the development of formal proofs by automating some aspects of it. 
\subsection{Coq Architecture}
Coq have two levels architecture - kernel and environment. A relatively small kernel based on a language with few primitive constructions (sorts, functions, inductive definitions, product types...) and a limited number of rules for type checking and computation. On top of the kernel, there is a rich environment to help designing theories and proofs. This environment offers mechanism like user extensible notations, tatics for proof automation, libraries... Any definition or proof defined in the environment is ultimately checked by the kernel so the environment can be used and extended safely.\\
As a Coq user, using high level constructions will help to solve a problem quickly. However, it might also important to understand the underlying low level language in order to develop new functionalities and to better control how certain constructions work. 
 
\chapter{Strand Space Formalization}


\chapter{Authentication Tests}

\chapter{Conclusion and Future Work}

% Let's assume this is the end of your thesis text.

% Now come appendices, if you had any.
% Appendices are automatically numbered, just like everything else in
% LaTeX. But only after you gave this command
\appendix

\chapter{More to say}

\section{A section within an appendix.}
This is an appendix.


% Last and least (at least, that's what the library says) - the
% Bibliography.


% you can save some space by having the bibliography singlespaced, if you want
\singlespacing

%
% You should become familiar with the BibTeX program, which
% uses a *.bib-file to collect all citations that you have. It's a lot
% prettier than typing all the citations right into the document. The 
% reference to citations also works well that way, but the exact 
% explanation of that will be on the CS-GSO homepage, whenever I'll ever 
% have time for that.
%
%
% If you use BibTeX, the bibliography is very easy. You refer to
% citations in the text with \cite{tag}, where tag is the tag that you
% defined in the bib-file.
% Then, you run bibtex once in a while during compilation, and the
% rest is done in two lines:



\bibliographystyle{plain}
\bibliography{mybib}



%============================

\end{document}









